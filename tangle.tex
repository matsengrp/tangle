\documentclass{amsart}

\usepackage{amsmath}
% \usepackage[german]{todonotes}

\newtheorem{theorem}{Theorem}[section]
\newtheorem{lemma}[theorem]{Lemma}
\newtheorem{corollary}[theorem]{Corollary}
\newtheorem{proposition}[theorem]{Proposition}
\newtheorem{example}[theorem]{Example}
\newtheorem{definition}[theorem]{Definition}
\newcommand{\RR}{{\mathbb R}}


\begin{document}
\title{Symmetries of pairs of phylogenetic trees}
\author[Matsen]{Frederick Matsen}
\address{}
\thanks{}

\date{\today}

%\begin{abstract}
%\end{abstract}

\maketitle

% % Make a todo table of contents.
% \makeatletter
% \providecommand\@dotsep{5}
% \makeatother
% \listoftodos\relax


\section{Introduction}
Pairs of trees on the same leaf set come up everywhere in phylogenetics.
Many people work on problems concerning pairs of trees.
For example SPR.
Pairs of trees on the same taxa are endowed with a natural symmetry, which can be formalized as an action of the symmetric group.

Moreover, all of these problems do not concern the actual leaf labels as such, but rather use the leaf labels as markers that can be used to map leaves of one phylogenetic tree on to another.
Due to the natural symmetries that exist in all phylogenetic trees, any given problem has an alternative phrasing with leaf labels moved in one or both of the trees.
Thus all such problems that are defined as such are actually defined in terms of this \textit{relative} leaf labeling.

This concept of a relative leaf labeling has been formalized as a \textit{tanglegram}.
There has been extensive work concerning tanglegrams, focusing on the problem of drawing them in a way that has minimum crossings.

In this paper we explore the mapping of pairs of leaf-labeled trees into the set of tanglegrams and explore the action of the symmetric group on the relatively labeled pairs of trees.


\section{Formalism}

\begin{definition}
A \textit{$n$-tanglegram}
\end{definition}



% \bibliographystyle{plain}
% \bibliography{tangle}
\end{document}
